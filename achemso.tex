
\documentclass[
%journal=ancac3, % for ACS Nano
%journal=acbcct, % for ACS Chem. Biol.
journal=jacsat, % for undefined journal
manuscript=article]{achemso}

\usepackage[version=3]{mhchem} % Formula subscripts using \ce{}
\usepackage{wasysym}
\usepackage{graphicx}
\usepackage{comment}


\newcommand*{\mycommand}[1]{\texttt{\emph{#1}}}

\author{Debjoy D Mallick}
\affiliation
{U.S. Army Research Lab, Aberdeen, USA}
\alsoaffiliation{Department of Mechanical Engineering, The Johns Hopkins University, Baltimore, USA}
\author{Meng Zhao}
\affiliation{Department of Mechanical Engineering, The Johns Hopkins University, Baltimore, USA}
\author{Brian Schuster}
\affiliation{U.S. Army Research Lab, Aberdeen, USA}
\author{K.T. Ramesh}
\email{ramesh@jhu.edu, 410-516-7735}
\affiliation{Department of Mechanical Engineering, The Johns Hopkins University, Baltimore, USA}
\alsoaffiliation{Hopkins Extreme Materials Institute, The Johns Hopkins University, Baltimore, USA}


\title[\texttt{achemso} demonstration]
{Photon Doppler Velocimetry in Plate Impact Experiments on Magnesium}

\begin{document}

\begin{abstract}
Normal and oblique plate impact experiments have long been the standard for generating controlled stress states at high strain rates in materials of interest. Thin specimens are impacted between an elastic anvil and impactor at various angles to develop different stress states. Dynamic strain rates ($10^4$ to $10^6 \:s^{-1})$ produce high energies as well, necessitating complex interferometry techniques to measure test results. The conventional techniques of normal velocity and transverse displacement interferometry (NVI and TDI) require careful multi-optic alignments, tedious polishing, and deposition of diffraction gratings. Here we implement photon doppler velocimetry (PDV), a methodology that uses common off the shelf fiber-optic telecommunications equipment, to simplify the diagnostic burden of these plate impact experiments as performed on AZ31B Magnesium samples. Thoughtful selection of probe angles allow characterization of longitudinal and shear waves in the specimen. We perform experiments with both NVI/TDI and PDV, comparing the advantages and disadvantages of both routes. 

\begin{comment}
Magnesium is attractive in applications where high specific strength is desirable but predominantly exhibits twinning as the major deformation mechanism due to asymmetry in hcp crystal structure. Data from the plate impact experiments can 
se experiments come with some difficulties,  
Magnesium offers a promising alternative material for structural components for

automobile, aerospace and other industries because of its high specific strength. However, the

asymmetric hcp crystal structure results in the activation of different deformation mechanisms

under different loading conditions, which leads to the anisotropy in the material response.

Although scattered data is available in the literature on the quasi-static and dynamic response,

very little is known about the constitutive response of magnesium and its alloys at strain rates

higher than 105

In this work, high strain rate pressure shear plate impact experiments with strain rates on the

order of 105 ~ 106

texture. Experiments are performed along different orientations of loading to study the effect of

anisotropy. The influence of strain rate on the strength, hardening and anisotropy is examined by

comparison with experimental data at lower rates (quasi-static and Kolsky bar). The deformation

mechanisms activated during the pressure-shear loading and the competition between different

mechanisms are also studied using post-mortem microscopy analysis.

-1

s

.

are performed on rolled AZ31B Magnesium alloy with a strong basal

-1

s

Magnesium (Mg) offers a promising alternative material for structural components in automobile, aerospace and other industries because of its high specific strength. However, the asymmetric hcp crystal structure cannot provide sufficient independent slip systems to accommodate homogeneous plastic deformation. As a result, twinning also plays a very significant role in the deformation of Mg. Different from dislocation slip, the activity of twinning is likely to be affected by both shear stress and normal stress hence leading to pressure sensitivity of the bulk response in Mg. 
We perform both pressure-shear plate impact and normal plate impact experiments on 100um-thick foil specimens to impose combined pressure-shear and hydrostatic pressure stress states, respectively. The material is rolled AZ31B Mg with an average grain size of 30 um. The occurrence of twinning and twin volume fraction are studied by analyzing the initial and post-mortem microstructures. We hope to isolate pressure effects on material properties by comparing the material behavior under these two different loading conditions.

\end{comment}

\end{abstract}
\newpage
\begin{figure}[H]
\centering
\includegraphics[scale=.65]{Capture.PNG}
\label{fig:1}
\end{figure}
\section{Bio}
Debjoy Mallick is a third year A.S.E.E. SMART Scholar in the Mechanical Engineering Department at the Johns Hopkins University. He is also a scientist at Army Research Lab at Aberdeen Proving Ground, Maryland. His interests lie in extremely high rate deformation of armor materials ranging through ceramics, metals, and composites. He holds a B.S. in Biomedical Engineering from the Johns Hopkins University.
%\bibliography{sample}


%We use this model to probe the rate-dependent strength, and compare these results against experimental measurements. Confinement effects are also considered. 

\end{document}
